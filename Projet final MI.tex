\documentclass[titlepage]{article}
\usepackage{graphicx}
\usepackage{babel} %[french]{babel} Ne fonctionne pas
\usepackage{hyperref}
\usepackage{amssymb}
\usepackage{amsthm}
\usepackage{amsmath}
\setcounter{tocdepth}{2}
\title{Le Théorème Fondamental de l'Arithmétique}
\author{Pierre VARNIER \\ Antoine GRENIER}
\date{18 janvier 2025}
\newtheorem{theorem}{Théorème :}
\newtheorem{exemple}{Exemple :}
\newtheorem{definiton}{Définition :}
\newtheorem{lemme}{Lemme :}

\begin{document}

\maketitle
\tableofcontents
\newpage
\section{Introduction}

Je ne peux pas faire de dépot sur git isima.
\\
\\
\hspace{1 cm} Le théorème fondamental de l'arithmétique prouve que tout entier supérieur ou égal à 2 possède une décomposition unique en facteurs de nombres premiers.

\section{Notation et définition}

\hspace{1 cm} Notons $\mathbb{N}$ l'ensemble des entiers naturels 
\begin{exemple}
$Des entiers naturels sont par exemples 0,1,2...$
\end{exemple}

\subsection{Définition nombre premier}

\begin{definiton}Nombres Premiers \\
Un nombre premier est un nombre qui ne peut être divisé que par lui-même et par 1.
\end{definiton}

\begin{exemple}
$Des exemples de nombres premiers sont 2,3,5,7...$
\end{exemple}

\begin{definiton}Le PGCD (Plus Grand Dénominateur Commun)\\
Le PGCD de 2 nombres est le plus grand entier naturel qui divise simultanément ces 2 nombres.

\begin{exemple}
$PGCD(24 ; 36) = 12 $
\end{exemple}
Grâce au PGCD, on peut donc trouver les diviseurs communs de $24$ et $36$, qui sont les diviseurs de $12$ : $1 ; 2 ; 3 ; 4 ; 6 ; 12$
\end{definiton}

\begin{definiton}Le PPCM (Plus Petit Multiple Commun)\\
Le PPCM de 2 nombres est le plus petit entier strictement positif qui soit multiple de ces deux nombres.

\begin{exemple}
$PPCM(16 ; 24) = 48$\\
$16 \times 3 = 48$	\\
$24 \times 2 = 48$ 	\\
\end{exemple}

\end{definiton}

\newpage

\subsection{Théorème fondamental de l'arithmétique}

\begin{theorem}
Tout entier naturel $\mathbb{N}$ $\geq 2$ peut être écrit de manière unique (à l'ordre des facteurs près) comme un produit de nombres premiers.
\end{theorem}

\subsection{Preuve}

\hspace{1 cm}On va d'abord démontrer l'existence d'une décomposition, puis son unicité.

\subsubsection{Existence}
\begin{lemme}
Pour démontrer l'existence, on peut utiliser une recurrence :
\begin{itemize}
\item Initialisation :
Pour n=2, qui est un nombre premier, la décomposition est lui même soit 2.
\item Recurrence :
\begin{itemize}
\item Hypothèse :
Supposons que $\forall k \in \mathbb{N} \in [2;n]$, on peut l'écrire comme un produit de nombres premiers.
\item Recurrence :
On veut montrer que $n+1$ peut aussi être écrit comme un produit de nombre premiers :
\item 1. Si $n+1$ est un nombre premier, alors il est déjà une décomposition en produit de nombres premiers, avec un seule facteur.
\item 2. Si $n+1$ n'est pas premier, alors il existe 2 entiers $a$ et $b$ tels que $n + 1 = a * b$, avec $2 \le a \le b < n+1$. Par hypothèse de recurrence, $a$ et $b$ peuvent être décomposés en produits de nombres premiers. En multipliant ces décompositions, on obtient alors une décomposition de $n+1$
\end{itemize}
\item Conclusion :
On a donc $\forall n \in \mathbb{N}, n > 1$, qui peut être écrit comme un produit de nombres premiers.
\end{itemize}
\end{lemme}

\subsubsection{Unicité}
\begin{lemme}
La preuve de l'unicité peut être obtenue à partir du lemme d'Euclide selon lequel, si un nombre premier p divise un produit ab, alors il divise a ou il divise b. Maintenant, prenons deux produits de nombres premiers qui sont égaux. Prenons n'importe quel nombre premier p du premier produit.
\end{lemme}

\subsection{Exemple d'une décomposition}
\begin{exemple}
$60$ peut se décomposer comme :
\begin{itemize}
\item $60 = 2 \times 30$
\item $60 = 2 \times 2 \times 15$
\item $60 = 2^2 \times 3 \times 5$
\\C'est la décomposition unique.
\end{itemize}

\end{exemple}
\section{Algorithmes}

\subsection{Algorithme de décomposition}
\begin{tabular}{l}
\hline
Algorithme de décomposition\\
\hline 
\end{tabular}
\begin{enumerate}
\item Entrer un entier $n \geq 2$
\item implémenter un i qui va de 2 à $\sqrt{n}$
\item Verifier n modulo $i == 0$
\begin{enumerate}
\item si n\% i == 0\\
$i$ sera un des facteurs premiers de $n$\\
$n$ prendra la valeur de $\frac{n}{i}$
$i$ prendra la valeur $2$\\
\item sinon\\
$i$ prendra la valeur de $i + 1$
\item si $i > sqrt{n}$ :\\
$n$ est un nombre premier et fera partie de la factorisation.
$n$ prendra la valeur de $\frac{n}{n} = 1$
\end{enumerate}
\item On reproduit les étapes précédentes jusqu'à avoir $n = 1$
\end{enumerate}

\subsection{Algorithme de PGCD}
\begin{tabular}{l}
\hline
Algorithme de PGCD\\
\hline 
\end{tabular}
\begin{enumerate}
\item On peut réutiliser l'algorithme de décomposition précédent sur 2 nombres $a$ et $b$.
\item On compare les valeurs de chacune des 2 décompositions.
\begin{enumerate}
\item Si il y a des valeurs en communs :\\
On gardera la plus grande valeur commune.\\
\item sinon\\
$PGCD( a ; b ) = 1 $
\end{enumerate}
\end{enumerate}

\subsection{Algorithme de PPCM}
\begin{tabular}{l}
\hline
Algorithme de PPCM\\
\hline 
\end{tabular}
\begin{enumerate}
\item Entrer 2 nombres $a$ et $b$.
\item Implémenter un $i$ dont les valeurs vont de $2$ à $b$ et un $j$ dont les valeurs vont de $2$ à $a$.
\item On compare chaque valeur de $b \times j$ et $a \times i$
\item On garde les valeurs communes et on retiens la plus petite c'est le PPCM.
\end{enumerate}

\section{Image}
Voici une image {d'écran}\ref{réf}.\\
begin{figure}[hbtp]\\
caption{Capture}\label{réf}\\
centering\\
includegraphics[scale=1]{.../../../../tmp/Captures d’écran/Capture d’écran du 2024-12-03}\\
end{figure}\\

\section{Référence}
\begin{itemize}
    \item \href{https://fr.wikipedia.org/wiki/Décomposition_en_produit_de_facteurs_premiers}{Page Wikipédia sur la décomposition en produit de facteurs premiers}.
\end{itemize}

\end{document}
